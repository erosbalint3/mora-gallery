\documentclass[12pt]{article}
\usepackage[a4paper, total={7in, 9in}]{geometry}

\usepackage[utf8]{inputenc}
\usepackage{t1enc}
\usepackage[magyar]{babel}

\usepackage{bera}

\usepackage{xcolor}
\colorlet{punct}{red!60!black}
\definecolor{background}{HTML}{EEEEEE}
\definecolor{delim}{RGB}{20,105,176}
\colorlet{numb}{magenta!60!black}
\usepackage{listings}
\lstset{
    language=bash, %% Troque para PHP, C, Java, etc... bash é o padrão
    basicstyle=\ttfamily\small,
    numberstyle=\footnotesize,
    numbers=left,
    backgroundcolor=\color{gray!10},
    frame=single,
    tabsize=2,
    rulecolor=\color{black!30},
    title=\lstname,
    escapeinside={\%*}{*)},
    breaklines=true,
    breakatwhitespace=true,
    framextopmargin=2pt,
    framexbottommargin=2pt,
    inputencoding=utf8,
    extendedchars=true,
    literate={á}{{\'a}}1 {í}{{\'i}}1 {é}{{\'e}}1,
}
\lstdefinelanguage{json}{
    basicstyle=\normalfont\ttfamily,
    numbers=left,
    numberstyle=\scriptsize,
    stepnumber=1,
    numbersep=8pt,
    showstringspaces=false,
    breaklines=true,
    frame=lines,
    backgroundcolor=\color{background},
    literate=
     *{0}{{{\color{numb}0}}}{1}
      {1}{{{\color{numb}1}}}{1}
      {2}{{{\color{numb}2}}}{1}
      {3}{{{\color{numb}3}}}{1}
      {4}{{{\color{numb}4}}}{1}
      {5}{{{\color{numb}5}}}{1}
      {6}{{{\color{numb}6}}}{1}
      {7}{{{\color{numb}7}}}{1}
      {8}{{{\color{numb}8}}}{1}
      {9}{{{\color{numb}9}}}{1}
      {:}{{{\color{punct}{:}}}}{1}
      {,}{{{\color{punct}{,}}}}{1}
      {\{}{{{\color{delim}{\{}}}}{1}
      {\}}{{{\color{delim}{\}}}}}{1}
      {[}{{{\color{delim}{[}}}}{1}
      {]}{{{\color{delim}{]}}}}{1},
}

\usepackage{hyperref}
\hypersetup{
    colorlinks=true,
    linkcolor=blue,
    filecolor=magenta,
    urlcolor=cyan,
    pdftitle={Overleaf Example},
    pdfpagemode=FullScreen,
    }

\date{2023.01.03.}
\title{Képek feltöltése a Móra Galériába}
\author{Kiss Ádám}

\begin{document}
\maketitle
\tableofcontents

\section{Bevezetés}
A Móra Galéria az SZTE Móra Ferenc Szakkollégium kép- és videoarchívumaként szolgál. Fontos szempont az egyszerű és minél több platformmal kompatibilis megjelenítő felület (\url{https://gallery.mora.u-szeged.hu/}), valamint egy egyszerű képfeltöltési lehetőség biztosítása.

\section{Megjeleítési képességek}
A galéria évszámokban és albumokban gondolkodik. Minden évhez több album tartozhat. Az albumnak az alábbiak lehetnek:
\begin{enumerate}
\item Kép (jpg,png,jpeg vagy JPG formátumban)
\item Videó (Böngésző kompatibilis videó és hangformátumban, mp4 tárolóban)
\item PDF fájlok (ezek nem kerülnek be az albumokba)
\item Egy borítókép (A leíró fájl tartalmazza)
\item Egy albumnév (A leíró fájl tartalmazza)
\item Egy albumelírás (A leíró fájl tartalmazza)
\item Egy leírófájl
\end{enumerate}
A leírófájl úgynevezett \textit{JSON} gép-nyelvtant követel meg formai követelményként. Példa egy leírófájlra:

\begin{lstlisting}[language=json]
{
"description": "Albumnev",
"thumbnail": "boritokep",
"longdescription": "Hosszabb leiras egy dobozban, a kepek mellett"
}
\end{lstlisting}
Amennyiben valamilyen leírást nem szeretnénk megadni, úgy az a sor kihagyható. Ügyelni kell viszont arra, hogy az utolsó felsorolt tulajdonság után ne legyen vesszó! A leírófájlokat a fényképek mellé kell elhelyezni \textit{description.json} néven. A hosszabb leírásba beszúrhatunk kis emotikonokat, de akár HTML formázást is.

A képek kis felbontásban nyilvánosságra kerülnek az interneten, illetve nagy felbontásban vízjelezve szintén nyilvánosságra kerülnek. A videókra nem kerül vízjel.

\section{Képek feltöltése}
Ahhoz, hogy valaki tudjon képet feltölteni, az egyik kollégiumi rendszergazdától engedélyt kell kérni, hogy a Fotós csoportba regisztrálja be az illető fiókját a kollégium Active Directory rendszerében.

\subsection{Fájlok elérése}
\subsubsection{Fájlok elérése Linux rendszerű számítógépeken}
Adjuk ki a következő parancsot:
\begin{lstlisting}
sudo mount -t cifs //media.mora.u-szeged.hu/photo-gallery  /mnt -o username=fnev@mora.u-szeged.hu,noperm
\end{lstlisting}
, ahol az \textit{fnev} szövegdarabot a felhasználónevünkkel helyettesítjük. Ezt követően a gyökérkönyvtár \textit{mnt} mappájában elérhetők lesznek a fájlok.

\subsubsection{Fájlok elérése Windows rendszerű számítógépeken}
\begin{enumerate}
\item Nyissuk le a Start + R billentyűkombinációt
\item Írjuk be a felugró ablakba a következő szöveget, majd nyomjunk Entert: \textit{\\\\media.mora.u-szeged.hu\\photo-gallery}
\end{enumerate}

\subsection{Fényképek felmásolása}
Válasszuk ki a \textit{Nyilvános} nevű mappát, majd ezt követően az évszámot. Hozzunk létre egy új mappát az albumnak, amennyiben még nem létezne. Ebbe a mappába másoljuk be a képeket. Mivel a weboldalon ábécé rendben jelennek meg a képek, ezért érdemes négyszámjegyű dátummal kezdeni a mappa nevét. Pl.: 0402 (Ez az április másodikát jelenti.).

A mappába elhelyezhetünk leíró fájlt is a korábban bemutatott minta alapján. Talán a legegyszerűbb átmásolni egy meglévő album leíróját, és azt módosdítani. A szerkesztéshez a Notepad++ program használata ajánlott.
\end{document}
